\documentclass[11pt, a4paper, sans]{moderncv}

\usepackage[margin=3cm]{geometry}
\usepackage{fontspec}

\setsansfont{PT Sans}

\moderncvtheme[black]{classic}
\renewcommand*{\emailsymbol}{}

\AtBeginDocument{\hypersetup{colorlinks=true, linkcolor=red}}

\firstname{Gleb}
\familyname{Shabanov}
\title{Software developer}
\address{Saint-Petersburg}{Russia}
\email{shabanov.ga@protonmail.com}
\homepage{https://github.com/suhr}

\begin{document}

\maketitle

\section*{Experience}
\cventry{2018--2019}{Junior software developer}{Zerho LLC}{Tox-rs}{}{
Documented the onion routing of the tox protocol, created ws-tox middleware and WebSocket based ws-tox-protocol for a Tox web client. \newline{} Maintained rstox bindings to c-toxcore.
}

\section*{Education}

\cventry{2016--2019}{Technical physics}{Peter the Great St. Petersburg Polytechnic University}{}{}{Incomplete}

\section*{Languages}

\cvitemwithcomment{Russian}{Native speaker}{}
\cvitemwithcomment{English}{Fair (upper intermediate)}{}

\section*{Software skills}

\cvitem{Good level}{Rust, Git}
\cvitem{Basic level}{Typescript, Python, Nix, HTML, CSS, C, Prolog, SQL, TLA+}

\section*{Pet projects and contributions}

\cvitem{Tox-rs}{There are PRs outside Zerho: \href{https://github.com/tox-rs/tox/pull/419}{migrating Tox-rs to an actual Tokio version}
and \href{https://github.com/tox-rs/tox/pull/420}{splitting Tox-rs into separate crates}.}

\cvitem{Razbor}{\href{https://github.com/razbor-rs/razbor}{Razbor} is a binary format specification language (WIP). It allows to specify a binary format in a formal way
and generate encoders and decoders for the format. Beside the compiler, there's also an incomplete specification of
the Tox protocol binary format: \href{https://github.com/razbor-rs/tox-razbor}{tox-razbor}.}

\cvitem{Blues}{\href{https://github.com/suhr/blues}{Blues} is a dialect Lambda Prolog (WIP). It is created as a tool for prototyping compiler passes for Razbor.
The name is inspired by an another language called \href{https://github.com/astampoulis/makam}{Makam}.}

\section*{Writings}

\cvitem{Conc}{There are several writings on the Conc programming language. \href{https://suhr.github.io/wcpl/intro.html}{WCPL} an overview of the idea.
Beside it, there are several incomplete articles (
    \href{https://suhr.github.io/papers/calg.html}{calg},
    \href{https://suhr.github.io/papers/cpat.html}{cpat},
    \href{https://suhr.github.io/papers/opcalg.html}{opcalg}
) on the theory. A basic type system for Conc is \href{https://github.com/suhr/makam-examples/}{described} using Makam language.}

\cvitem{Gfx-rs}{\href{https://suhr.github.io/gsgt/}{Graphics by Squares} is a tutorial on the pre-ll gfx-rs.}

\end{document}
